\documentclass{ctexart}
\usepackage{amsmath}
\usepackage{booktabs}
\usepackage{minted}
\usepackage{url}
\newcommand{\bx}{\boldsymbol{x}}		%粗体x

\ctexset {
section = {
	format = \large\bfseries\raggedright,
	name = {,、},
	number = \chinese{section},
	aftername = {},
},
subsection = {
	format = \normalsize\bfseries,
	name = {,.},
	number = \arabic{subsection},
	indent = {2em}
},
subsubsection = {
	format = \normalsize\bfseries,
	name = {(,)},
	number = \arabic{subsubsection},
	indent = {2em}
}
}

\title{\bfseries 模式识别大作业报告}
\author{\kaishu 高童 \quad 计43 \quad 2014011357}
\date{}

\begin{document}
\maketitle
\section{问题描述}
本次大作业是一个手写数字识别问题,是一个经典的分类问题。输入图像是$32 \times 32$的黑白手写数字图像,目标是将其分类到$0\sim 9$的十类中。

根据题目提示,实验要点包括:
\begin{itemize}
	\item 构造受限玻尔兹曼机RBM,以图像为输入,对输出单元用传统分类器进行分类;
	\item 图像格式不一,因此需要进行预处理;
	\item 由于原始图像有$32 \times 32=1024$个像素,将全部像素作为输入单元会导致输入单元数量过多,RBM训练过慢,因此需要通过预处理降低输入单元数量。
\end{itemize}

\section{代码设计}
\subsection{整体设计}
整体代码用Python实现。

代码预处理部分的框架如下。首先是读入图片,分别经过的函数分别包括:
\begin{enumerate}
	\item \texttt{Image.open()}。用PIL库打开图片。
	\item \texttt{np.array()}。将打开的图片转换为\texttt{numpy}格式多维数组。
	\item \texttt{singleTunnel()}。如果图片有超过1个通道,将所有通道上的值平均,返回一个灰度图像。
	\item \texttt{compressImg()}。降低输入单元的函数。
\end{enumerate}
示例代码如下:
\begin{minted}{Python3}
# load second directory
trainDigitPath = trainPath + 'hjk_picture'
p = Path(trainDigitPath)
for f in p.iterdir():
    if f.suffix in ['.png', '.jpg']:
        print(f.name)
        x = np.array(Image.open(f))
        y = compressImg(singleTunnel(x))
        I.append(x)
        X.append(y)
        if f.name[1] == '_':
            Y.append(int(f.name[0]))
        else:
            Y.append(int(f.name[2]))
\end{minted}

具体分类和测试的部分框架如下\footnote{参考Scikit-learn网站的教程,标题为Restricted Boltzmann Machine features for digit classification,网址\url{http://scikit-learn.org/stable/auto_examples/neural_networks/plot_rbm_logistic_classification.html}。}。首先读入训练数据和标签,并将训练数据归一化到0—1区间。
\begin{minted}{Python3}
X_train = digits['data']
Y_train = digits['target']
X_train = (X_train - np.min(X_train, 0)) / (np.max(X_train, 0) + 0.0001)
\end{minted}

随后构造一个Pipeline,设置相关参数,并在其上进行训练:
\begin{minted}{Python3}
rbm = BernoulliRBM(random_state=0, verbose=True)
logistic = linear_model.LogisticRegression()
classifier = Pipeline(steps=[('rbm', rbm), ('logistic', logistic)])

# Hyper-parameters.
rbm.learning_rate = 0.04
rbm.n_iter = 20
rbm.n_components = 10000
logistic.C = 5000.0

print("Training")
classifier.fit(X_train, Y_train)
\end{minted}

最后,利用scikit-learn自带的函数生成测试报告:
\begin{minted}{Python3}
print()
print("Logistic regression using RBM features:\n%s\n" % (
    metrics.classification_report(
        Y_test,
        classifier.predict(X_test))))
\end{minted}

\subsection{降低输入单元}
为了降低RBM的输入单元数量,采用的策略如下:将图像在横、竖两个方向上
分别隔一个像素取一个像素,这样图像大小就变味原来的四分之一,
总共只剩$16 \times 16=256$个像素。在后来的测试中,可以发现
这样不会降低分类的准确率。

相关函数代码如下:
\begin{minted}{Python3}
def compressImg(x):
    [r, c] = [32, 32]

    comp = np.empty([math.floor(r/2), math.floor(c/2)])
    for i in range(0, comp.shape[0]):
        for j in range(0, comp.shape[1]):
            comp[i, j] = x[2*i, 2*j];

    return comp.flatten()
\end{minted}

\section{测试结果}
在给定的训练集和测试集上,分类效果比较好。
此处着重讨论降维程度在不同RBM规模下对分类效果的影响。

在限定参数:学习率=0.04,循环次数=20的情况下,分别改变输入单元压缩的程度
和隐藏单元的数量,观察分类正确率和所耗时间,结果列于表\ref{1}中。

\begin{table}[htbp]
	\centering
	\caption{各个参数下的正确率和程序运行时间}
	\label{1}
	\vspace{0.5em}
	\begin{tabular}{ccc}
		\toprule
		输入参数 & 正确率 & 所耗时间 \\
		\midrule
		1024维,内单元3000 & 97\% & 149.8秒\\
		256维,内单元3000 & 98\% & 47.8秒\\
		100维,内单元3000 & 78\% & 26.3秒\\
		\midrule
		256维,内单元10000 & 100\% & 165.2秒\\
		100维,内单元10000 & 83\% & 92.6秒\\
		\bottomrule
	\end{tabular}
\end{table}

可以看出,在图像大小变为原来1/4时,模型训练时间明显减少,
但正确率没有下降。
在运行时间相同的情况下,1/4大小输入时,内单元数可以增加到约10000,
对模型有更好的拟合能力。
但当图像大小变为原来的约1/9时,由于原图信息丢失,准确率下降较多。

综上可知,每次图像都缩大小为原来1/4可以作为类似数据集上的标准预处理程序。
在不同的训练集上,都可以做类似的测试,寻找缩小原图的最好比例。

\section{实验结论和收获}
RBM是理论上可以达到全局最优的一种模型。
但由于RBM采用类似模拟退火的随机策略,收敛比较慢。
为加速RBM收敛,采用合适方法降低输入单元数是必要的。

本次实验中,我亲身体会了降低输入单元数对分类器性能的巨大改善,
加深了对RBM的理解,也
从改善分类器中获得了成就感。感谢陶老师和助教这一学期的指导。

\end{document}
